% Hauptteil

\section{Simulation des Batteriespeichersystems}

In diesem Kapitel werden die wichtigsten Simulationsschritte dargestellt und erläutert. Weiterhin erfolgt eine Betrachtung der Ergebnisse.

\subsection{Parametervariation und Simulationsziele}

Die Parametervariation soll dazu dienen, möglichst viele Fälle möglichst genau darstellen zu können. Folgende Parameter fließen modellendogen in die Durchläufe der Simulation ein:

\begin{itemize}
\itemsep-0.5em
	\item Der Hausverbrauch von \SIrange{3000}{10000}{\kwh}
	\item Die Kapazität der Batterie von \SIrange{8}{16}{\kwh}
	\item Die Größe der Photovoltaikanlage von \SIrange{5.5}{10}{\kwp}
	\item Die EEG-Vergütung für die Stromeinspeisung von \SIrange{9.87}{12.75}{\ctkwh}
	\item Der Grundpreis des Vergleichstromtarifs von \SIrange{5}{12}{\Eurkwh}
	\item Der Arbeitspreis des Vergleichstromtarifs von \SIrange{25}{32}{\ctkwh}
\end{itemize}

\noindent Weiterhin sind folgende Parameter modellexogen vorgegeben und nicht variiert:

\begin{itemize}
\itemsep-0.5em
	\item Die nominale AC-Leistungsaufnahme des Batteriewechselrichters \SI{3.3}{\kw}
	\item Die nominale AC-Leistungsabgabe des Batteriewechselrichters \SI{3.3}{\kw}
	\item Der mittlerer Umwandlungswirkungsgrad des Batteriewechselrichters im Ladebetrieb \SI{94.4}{\percent}
	\item Der mittlerer Umwandlungswirkungsgrad des Batteriewechselrichters im Entladebetrieb \SI{94.5}{\percent}
	\item Der mittlerer Umwandlungswirkungsgrad des Batteriespeichers \SI{93.8}{\percent}
	\item Die präqualifizierte Leistung des virtuellen Kraftwerks $\pm \SI{1}{\mega\watt}$
	\item Die theoretische maximale Leistung des virtuellen Kraftwerks $\pm \SI{1.98}{\mega\watt}$
	\item Der sonnenBonus in Höhe von \SI{0.25}{\ctkwh}
\end{itemize}

\noindent Das Ziel der Simulation ist es, in erster Linie einen Kostenvergleich der Vertragsvarianten für den Kunden zu ermöglichen. Die wichtigsten zu ermittelten Größen sind von daher die jährlichen Kosten. Außerdem soll ermittelt werden, wie stark die Batterie durch das virtuelle Kraftwerk mehr belastet wird. Hierfür werden die auftretenden Vollzyklen pro Jahr berechnet.

\subsection{Simulation des Einflusses des virtuellen Kraftwerks}

Dieses Kapitel soll die Simulation des Einflusses des virtuellen Kraftwerks auf das eigenverbrauchsoptimierte Verhalten der Batterie erläutern. Weiterhin soll eine Bewertung der Approximation vorgenommen werden.

\subsubsection{Erläuterung der Simulation}

Die Erbringung von Primärregelleistung hat immer Vorrang vor der Eigenverbrauchsoptimierung des Kunden. Somit muss ermittelt werden, in welchen Zeitschritten es zu einer Erbringung von Primärregelleistung kommt.\medskip\\
Als erste Approximation wird angenommen, dass das virtuelle Kraftwerk an \SI{80}{\percent} der Tage des Jahres an der Erbringung von Primärregelleistung teilnimmt. Hierdurch werden Wartung und nicht erfolgreiche Ausschreibungen abgedeckt. Um einen ausreichenden Ladestand zu garantieren, wird weiterhin zwischen zwei Zuständen unterschieden:

\begin{enumerate}
\itemsep-0.5em
	\item Die Ladestandsgrenzen sind aktiviert
	\item Die Erbringung von Primärregelleistung und die Ladestandsgrenzen sind aktiviert
\end{enumerate}

\noindent Der erste Fall tritt immer dann auf, wenn auf einen Tag ohne Erbringung von Primärregelleistung ein regelleistungsaktiver Tag folgt. In diesem Fall werden die Ladestandsgrenzen von \SIrange{20}{80}{\percent} bereits \SI{12}{\hour} vor der eigentlichen Erbringung aktiviert. Damit soll möglichst sichergestellt werden, dass zu Beginn der Regelleistungserbringung genügend Energie in den Speichern zur Verfügung steht. Weiterhin soll auf diese Weise auf eine Simulation des Nachlademanagements des virtuellen Kraftwerks verzichtet werden können.\medskip\\
Im nächsten Schritt, muss ermittelt werden ob die eigene Batterie in dem vorliegenden Zeitschritt an der Erbringung von Primärregelleistung teilnimmt. Hierfür wurde ein einfacher boolean Minuten-Vektor geschaffen, mit folgenden Bedeutungen:

\begin{itemize}
\itemsep-0.5em
	\item $0 =$ Eigenverbrauchsoptimierung
	\item $1 =$ Erbringung von Primärregelleistung
\end{itemize}

\noindent Damit die Batterie an der Regelleistungserbringung teilnimmt, muss die Regelleistungserbringung des gesamten virtuellen Kraftwerks aktiv sein, die Frequenzabweichung der Netzfrequenz außerhalb des Totbandes liegen und die Batterie zu den verwendeten Batterien des Zeitschritts gehören.\\
Um die letzte Vorraussetzung zu approximieren, wurde vorerst die Wahrscheinlichkeit berechnet, dass die Batterie Primärregelleistung in dem Zeitschritt erbringen muss.

\begin{equation}
\label{eq:approx_FCR}
	p_{\text{FCR}}\left(t\right) = \left|P_{\text{FCR}_{\text{VPP}}}\left(t\right)\right| \cdot \frac{P_{\text{PQ}}}{P_{\text{max}}}
\end{equation}

\begin{conditions}
	t															&		Zeitschritt\\
	p_{\text{FCR}}\left(t\right)					&		Wahrscheinlichkeit der Regelleistungserbringung\\
	P_{\text{FCR}_{\text{VPP}}}\left(t\right)	&		Abgerufene Primärregelleistung des virtuellen Kraftwerks im Zeitschritt\\
	P_{\text{PQ}}										&		Präqualifizierte Leistung des virtuellen Kraftwerks\\   
	P_{\text{max}}									&		Theoretische maximale Leistung des virtuellen Kraftwerks \\   
\end{conditions}

\noindent Der hierbei entstehende Vektor wird anschließend mit einem Vektor verglichen, dessen Variablen zufällig in einem Bereich von \SIrange{0}{100}{\percent} generiert wurden. Liegt der Wert der Approximation oberhalb der Zufallsvariable, wird in diesem Zeitschritt Primärregelleistung erbracht.\\
Dies gilt jedoch nur, wenn zeitgleich auch das virtuelle Kraftwerk aktiv ist. Werden die beiden Vektoren miteinander abgeglichen, ergibt sich, dass die Regelleistungserbringung der Batterie ca. \SI{3}{\percent} der Gesamtzeit ausmacht.

\subsubsection{Bewertung der Approximation}

Um die Approximation bewerten zu können, muss ein Optimum definiert werden. In der Simulation wird von einem homogenen virtuellen Kraftwerk ausgegangen. Das heißt, dass die Last genau gleichmäßig zwischen den Batterien aufgeteilt wird. Die theoretische Regelenergie berechnet sich aus den Lastgängen der Netzfrequenz und dem Aktivitätsvektors des virtuellen Kraftwerks wie folgt:

\begin{code}
\captionof{listing}{Berechnung der theoretischen Regelenergie der Batterien}
\label{code:VPP}
\begin{minted}{matlab}
% Berechnung der theoretischen Energieabgabe des Batteriesystems
% für die Erbringung von FCR

% Nur berechnen, wenn VPP auch aktiv
Pvppactive = LProf.pvpp .* LProf.vppactive;

% Negative Regelleistung - Batterie laden
E_neg = abs(sum(Pvppactive(Pvppactive < 0))) * 1000 / 60 / VPP.n_Bat;

% Positive Regelleistung - Batterie laden
E_pos = sum(Pvppactive(Pvppactive > 0)) * 1000 / 60 / VPP.n_Bat;
\end{minted}
\end{code}

\noindent Bei einer homogenen Aufteilung der Last bedeutet dies die Erbringung von \SI{477.1}{\kwh} negativer (Batterieladung) und \SI{392.6}{\kwh} positiver (Batterieentladung) Primärregelleistung.

\subsubsection*{Negative Primärregelleistung}

Das Ergebnis der Simulation zeigt, dass die erbrachte negative Regelleistung bei \linebreak \SI[multi-part-units = single]{487.0(10)}{\kwh} liegt. Somit wird die negative Regelleistungserbringung in der Simulation leicht überbewertet.

\subsubsection*{Positive Primärregelleistung}

Im Mittel liegt die erbrachte positive Regelleistung bei \SI[multi-part-units = single]{390.1(10)}{\kwh}. Die Erbringung positiver Regelleistung wird in der Simulation leicht unterbewertet dargestellt. 

\subsubsection*{Einordnung der Ergebnisse}

Die Abweichung von der idealen Aufteilung der Last beträgt \SI{2.1}{\percent} bei der Erbringung von negativer Regelleistung und \SI{0.6}{\percent} bei der Erbringung von positiver Regelleistung. Somit ist die Approximation als sehr gut einzuschätzen.

\subsection{Kritik an der Approximation}

% Kritik: keine differentation bzgl Speichergröße
% kein nachladen wenn SOC zu gering --> Bild!!

\newpage