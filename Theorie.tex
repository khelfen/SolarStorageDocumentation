% Theoretische Grundlage, Berechnungsmethodik und Datensätze

\section{Theoretische Grundlagen und Datengrundlage}

In diesem Kapitel sollen die wichtigsten Grundlagen erläutert werden, um eine Bewertung des Einflusses der Nutzung des Speichersystems für die Erbringung von Primärregelleistung vornehmen zu können. Hierzu zählen die theoretischen Grundlagen der Erbringung von Primärregelleistung und des virtuellen Kraftwerks inklusive der verwendeten Hardware. Weiterhin wird auf die wichtigsten verwendeten Python und Matlab Befehle, sowie die verwendeten Datensätze und deren Aufarbeitung eingegangen. Zusätzlich wird das Vertragsmodell der sonnenFlat erläutert, um die Kostenstruktur darstellen zu können.

\subsection{Primärregelleistung}

Systemdienstleistungen werden eingesetzt, um die Versorgungssicherheit des europäischen Verbundsystems sicherzustellen. Da der Großteil dieser Maßnahmen bisher von konventionellen Kraftwerken erbracht wird, ist, neben dem Zubau von regenerativen Energieanlagen, das Erschließen von alternativen Systemdienstleistungserbringern ein essenzieller Bestandteil der Energiewende. In dieser Simulation soll im Näheren die Regelleistung und insbesondere die Primärregelleistung betrachtet werden.\medskip\\
Regelleistung wird genutzt, um die Netzfrequenz zu stabilisieren. Primärregelleistung stellt dabei die erste Instanz der Frequenzregelung dar und dient dazu momentane Unterschiede zwischen Leistungsangebot und -nachfrage im Stromnetz auszugleichen. Das durch den Übertragungsnetzbetreiber definierte Totband beträgt $\pm \SI{10}{\milli\hertz}$. Wird die Soll-Netzfrequenz von \SI{50}{\hertz} um mehr als das Totband über- oder unterschritten wird automatisch begonnen Regelleistung zu erbringen, die dabei proportional mit der Frequenzabweichung steigt. Bei einer Differenz von $\pm \SI{200}{\milli\hertz}$ ist allerdings ein kritischer Wert erreicht und die volle präqualifizierte Regelleistung des Kraftwerks muss erbracht werden. \parencite{nextKW20}.


\subsection{sonnenBatterie eco 8.0}\label{sec:SB_eco8_Theorie}

Die physische Grundlage des virtuellen Kraftwerks wird durch eine Flotte von Heimspeichern gebildet. In dieser Simulation werden Heimspeicher der sonnenBatterie eco 8.0 Serie gewählt. Die Möglichkeiten und Limitationen des virtuellen Kraftwerks richten sich entsprechend nach den technischen Eigenschaften dieser Lithium-Eisenphosphat Akkumulatoren.\\
Die einzelnen Batterien besitzen je nach Ausstattung eine nutzbare Batteriekapazität von \SIrange{4}{16}{\kwh}. Vereinfachend wird angenommen, dass nur sonnenBatterien mit einer Kapazität von mindestens \SI{8}{\kwh} eingesetzt werden. Ab einer Kapazität von  \SI{8}{\kwh} ist jede Batterie  mit einem Wechselrichter ausgestattet, der eine Nennleistung von \SI{3.3}{\kilo\watt} besitzt \parencite{sonnenBat18}.\\
Der mittlere Wirkungsgrad des Wechselrichters beträgt im Entladefall \SI{94.5}{\percent} und im Ladefall \SI{94.4}{\percent}. Weiterhin weißt die Batterie einen Wirkungsgrad von \SI{93.8}{\percent} auf \parencite{htwInspek19}.

\subsection{Das virtuelle Kraftwerk}\label{sec:VK_Theorie}

Das virtuelle Kraftwerk der Simulation wurde mit einer Gesamtleistung von \SI{1}{\mega\watt} präqualifiziert. Als Annahme wurden hierzu insgesamt 600 Heimspeicher der sonnenBatterie eco 8.0 Serie vernetzt, mit einer Gesamtleitung von \SI{1.98}{\mega\watt}. Der theoretische Leistungsspielraum des Kraftwerks liegt somit deutlich über der präqualifizierten Leistung.\\
Dieser Umstand wird zum einen gerechtfertigt durch die Tatsache, dass nicht garantiert werden kann, dass jede Batterie zu jedem Zeitpunkt verfügbar ist. Es kann zu Störungen in Hard- und Software der Speicher kommen, aber auch die Internetverbindung kann zeitweise unterbrochen werden. Außerdem unterliegt die Kontrolle der Anlage in erster Linie der Privatperson und nur bedingt dem Betreiber.\\
Die am stärksten überwiegenden Einflussgrößen auf die Leistungsfähigkeit des virtuellen Kraftwerks sind aber der eigenverbrauchsoptimierte Betrieb der Anlage und der Ladestand. Je nach dem aktuellem Betriebspunkt kann nur noch begrenzt zusätzlich Primärregelleistung aufgebracht werden, bis die maximale Umrichterleistung erreicht wird. Bei einem zu hohen bzw. zu niedrigen Betriebspunkt der Batterieflotte kann also unter Umständen die geforderte Regelleistung nicht geliefert werden. Ein zu hoher bzw. niedriger Ladestand beschränkt zudem den Zeitraum, über den Regelleistung erbracht werden kann. Diesen Faktoren muss im Realbetrieb entgegengewirkt werden mit aktiven Lademanagement und Ladezustandsgrenzen, um zu gewährleisten, dass die präqualifizierte Regelleistung zu jeder Zeit geliefert werden kann.\\
Das grundlegende Funktionsprinzip des virtuellen Kraftwerks wird durch einen übergeordneten Regler bestimmt. Dieser wird in der Simulation stark vereinfacht. So wird von einem homogenen Verhalten der Batterien ausgegangen. Dies bedeutet, dass jede Batterie  die gleichen technischen Parameter besitzt und das sich die zugehörigen Photovoltaikanlagen ebenfalls identisch verhalten. Das führt dazu, dass den einzelnen Batterien feste Ladestandsgrenzen zugeordnet werden können, damit die Erbringung von Regelleitung im Bedarfsfall gewährleistet werden kann.

\begin{figure}[H]
	\begin{center}
		\includegraphics[width=\textwidth]{Bilder/P-E_factor.png}
		\caption{Zulässiger Arbeitsbereich bei der Erbringung von Primärregelleistung \parencite[s. S. 61][]{regel_PQ19}}
		\label{fig:P_E_factor}
	\end{center}
\end{figure}

\noindent Im Falle begrenzter Energiespeicher kommen die in Abbildung \ref{fig:P_E_factor} dargestellten zulässigen Arbeitsbereiche zur Anwendung. Durch dies wird sichergestellt, dass der Energiespeicher jederzeit seine vollständige angebotene Regelleistung für \SI{15}{\minute} zur Verfügung zu stellen.\medskip\\
In dem Fall des simulierten virtuellen Kraftwerks besteht ein Speicherverhältnis von \SIrange{4.8}{9.6}{}, je nach Größe der einzelnen Speichereinheiten. Da jedoch das Lademanagement innerhalb dieser Simulation nicht abgebildet werden kann, wurde sich für Ladestandsgrenzen von \SI{80}{\percent} im oberen und \SI{20}{\percent} im unteren Energiebereich entschieden.

\subsection{Die sonnenFlat}

Das Konzept der sonnenFlat beruht in erster Linie auf der sogenannten Freistrommenge. Diese dient als Leistungstausch für die Einschränkungen des eigenverbrauchsoptimierten Verhaltens der Batterie. Die Freistrommenge bezieht sich auf den Gesamtstromverbrauch und beinhaltet den Direktverbrauch der Photovoltaikanlage und den zusätzlichen Netzbezug. Der Netzbezug der Batterie für die Erbringung von Primärregelleistung wird getrennt bilanziert.

\begin{figure}[H]
	\begin{center}
		\includegraphics[width=\textwidth]{Bilder/flat.png}
		\caption{Varianten der sonnenFlat \parencite[s. S. 26][]{flat20}}
		\label{fig:flat_ver}
	\end{center}
\end{figure}

\noindent In Abbildung \ref{fig:flat_ver} sind die verschiedenen Varianten der sonnenFlat vorgegeben. Die Freistrommenge des Kunden hängt sowohl von der Leistung der Photovoltaikanlage und der Kapazität des Batteriespeichers ab. Die minimale Größe der Photovoltaikanlage beträgt hierbei \SI{5.5}{\kwp}, weshalb dies als minimale Größe der Simulation angenommen wurde. In dieser Simulation wird davon ausgegangen, dass der Kunde immer die für seine spezielle Situation größtmögliche sonnenFlat erhält. Hiervon ausgenommen ist die sonnenFlat 8000, da bei dieser höhere Community-Beiträge anfallen. Bei Überziehung der Freistrommenge von bis zu \SI{2000}{\kwh}, fallen bei der sonnenFlat Arbeitspreise von \SI{23}{\ctkwh} und darüber \SI{25.9}{\ctkwh} an \parencite{flat20}. Der Break-even-Point der sonnenFlat 8000 gegenüber der sonnenFlat 6750 ist somit ab einem jährlichen Stromverbrauch von \SI{7272}{\kwh} erreicht.\medskip\\
Eine weitere Besonderheit des sonnenFlat Vertrages ist der sogenannte sonnenBonus.. Hierbei handelt es sich um einen Aufschlag von  \SI{0.25}{\ctkwh} für jede eingespeiste Kilowattstunde der Photovoltaikanlage auf die übliche EEG-Vergütung \parencite{sonnenBonus20}.

\subsection{Verwendete Python und Matlab Befehle}

In diesem Kapitel werden die wichtigsten Python und Matlab Befehle erklärt. Auf Befehle, die bereits Teil der Vorlesung waren, wird nicht eingegangen.

\subsubsection*{Python}

\textbf{Glob}\\
Das \texttt{Glob}-Modul findet alle Pfadnamen, die mit einem angegebenen Muster übereinstimmen \parencite{PyGlob20}.\medskip\\

\noindent\textbf{Pandas}\\
\texttt{Pandas} bietet Funktionen für die Strukturierung und Analyse von Datensätzen \parencite{PyPan20}. Die wichtigsten verwendeten Befehle der \texttt{Pandas}-Bibliothek lauten wie folgt:

\begin{conditions}
	\text{\texttt{read\_csv()}}		&		Einlesen einer CSV-Datei als Datenfeld\\
	\text{\texttt{DataFrame()}}	&		Erstellt ein Datenfeld mit gewünschten Parametern\\
	\text{\texttt{multiply()}}		&		Ermöglicht Multiplikation eines Datenfeldes mit einem Skalar\\
	\text{\texttt{round()}}			&		Runden der Elemente eines Datenfeldes\\
	\text{\texttt{to\_csv()}}			&		Speichern eines Datenfeldes als CSV-Datei\\
	\text{\texttt{concat()}}			&		Verketten von einzelnen Datenfeldern\\
\end{conditions}


\subsubsection*{Matlab}

\textbf{Polyval}\\
Der Befehl \texttt{polyval(p, x)} wertet ein Polynom \texttt{p} an jedem Punkt in \texttt{x} aus. Das Argument \texttt{p} ist ein Vektor der Länge \texttt{n $+$ 1}, dessen Elemente die Koeffizienten (in absteigenden Potenzen) eines Polynoms \texttt{n}-ten Grades sind \parencite{MatPol20}:

\[ p\left(x\right)	= p_1 \cdot x^n + p_2 \cdot x^{n-1} + ... + p_n \cdot x + p_{n+1}	\]

\noindent\textbf{Reshape}\\
\texttt{Reshape} formt ein gegebenes Array in eine gewünschte Form um. Beispielsweise formt \texttt{reshape(A, [2,3])} \texttt{A} in eine 2-mal-3-Matrix um \parencite{MatRe20}.

\subsection{Datengrundlage}

Für die Simulation des Einflusses des virtuellen Kraftwerks kamen drei Datensätze zum Einsatz. Hierzu zählt ein Haushaltslastprofil, das Erzeugungsprofil einer Photovoltaikanlage und das Profil der Netzfrequenz im europäische Verbundsystem, welche modellexogen die Grundlage der Simulation bilden. Jeder Datensatz hat eine 1-minütige Auflösung und bildet ein ganzes Jahr ab. Die Aufarbeitung der einzelnen Datensätze erfolgt mit Hilfe von Python.

\subsubsection{Erzeugungsprofil der Photovoltaikanlage}

Das Erzeugungsprofil der Photvoltaikanlage wurde innerhalb der Vorlesung zur Verfügung gestellt. Dieses findet auch hier Anwendung und ist in der Datei \texttt{A04\_Daten.mat} hinterlegt. In der Variable \texttt{ppvs} ist die spezifische AC-Leistungsabgabe des Photovoltaik-Systems, normiert auf die nominale Photovoltaik-Generatorleistung hinterlegt.

\subsubsection{Haushaltslastprofil}

Das Haushaltslastprofil entspricht einem repräsentativen elektrische Lastprofile für Wohngebäude in Deutschland auf 1-minütiger Datenbasis. Dieses wird durch die Hochschule für Technik und Wirtschaft Berlin zur Verfügung gestellt \parencite{htwLast18}.\\
Verwendet wurde das dritte Lastprofil der Datei \texttt{CSV\_74\_Loadprofiles\_1min\_W\_var(1).zip}. Auch dieses wurde normiert. Dafür wurde zuvor die maximale Leistungaufnahme des Systems bestimmt und mit dieser die spezifische Leistungsaufnahme des Systems zu jeder Minute ermittelt.\medskip\\
Realisiert wurde dies mit folgendem Code:

\begin{code}
\captionof{listing}{Aufbereitung des Datensatzes des repräsentativen elektrischen Lastprofils für Wohngebäude}
\label{code:household}
\begin{minted}{python}
import pandas as pd

# Namen für die Spalten des Datensatzes bestimmen
names = list()
for i in range(74):
    names.append('H{}' .format(i+1))
    
# Einlesen der Daten
df_Load1 = pd.read_csv('DataRaw\\PL1.csv', names=names, sep=',')

# Ziel Datensatz isolieren und normieren
df_Household = pd.DataFrame()
df_Household['P_H'] = df_Load1.H3.multiply(1/max(df_Load1.H3))

# Speichern als .csv und runden
df_Household.round(5).to_csv('P_H.csv')
\end{minted}
\end{code}

\subsubsection{Profil der Netzfrequenz}

Das Profil der Netzfrequenz im europäische Verbundsystem liegt in 1-sekündiger Auflösung monatsweise für das Jahr 2018 vor. Zur Verfügung gestellt wurden die entsprechenden Datensätze durch Herrn Dipl.-Ing. (FH) Markus Jaschinsky \parencite{Jaschinsky18}.\medskip\\
Ziel der Aufarbeitung war es die einzelnen Profile zusammenzuführen und in eine 1-minütige Auflösung umzuwandeln. Anschließend sollte aus diesem Profil der Lastgang des virtuellen Kraftwerks, normiert auf die ausgeschriebene Primärregelleistung, ermittelt werden. Dieses erfolgte auf Grundlage des folgenden Codes:

\begin{code}
\captionof{listing}{Aufbereitung der Datensätze des Profils der Netzfrequenz im europäische Verbundsystem}
\label{code:VPP}
\begin{minted}{python}
import glob
import pandas as pd

# Namen der einzelnen Datein ermitteln
lst_csv = glob.glob("2018\*.csv", recursive=True)
# Spaltennamen festlegen
names = ['fq', 'delete']

# Ergebnisliste vorinitialisieren
lst_df = [0]*len(lst_csv)
i = 0

# Einlesen der einzelnen Datensätze
for name in lst_csv:
    lst_df[i] = pd.read_csv(name, names=names, sep=';')
    lst_df[i].index.name = 'ts'
    lst_df[i] = lst_df[i].drop(['delete'], axis=1)
    i += 1
    
# Zusammenführen der Monatsdatensätze
df_fq18 = pd.DataFrame()

for n in range(len(lst_df)):
    df_fq18 = pd.concat([df_fq18, lst_df[n]], sort=False)
    
# Umwandeln in 1-minütige Auflösung
df_fq18_minutes = df_fq18.copy().iloc[::60, :]

# Umrechnen der Netzfrequenz
# in die Sollvorgabe der Leistungserbringung des VPP
df_fq18_minutes['p_VPP'] =
[0 if 49.99 < fq < 50.01 else (50 - fq)*10/2 for fq in df_fq18_minutes.fq]

# Speichern als .xlsx
df_fq18_minutes.round(3).to_excel('P_VPP.xlsx')
\end{minted}
\end{code}

\newpage