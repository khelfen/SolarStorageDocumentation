% Shortcuts for SI Values
% Grundlagen: https://www.namsu.de/Extra/pakete/Siunitx.html
% Gute Erklärung der Shortcuts: https://texwelt.de/fragen/2588/wie-schreibe-ich-zahlen-mit-einheiten-richtig

%\DeclareSIUnit{\BeladungsDichte}{\kilo\gram_{\textup{H\textup{2}}\per\kilo\gram_{\textup{FeTi}}}}		% Beladungsdichte
%\DeclareDocumentCommand\BeladungsDichte{O{}m}{\SI[#1]{#2}{\BeladungsDichte}}

%\DeclareSIUnit[]\NormVolumen
%{\text{\ensuremath{\cubic\meter_{\textup{i.N.}}}}}

%%%%%%% New SIValues

\DeclareSIUnit\kgh{\kg_{H\textsubscript{2}}}
\DeclareSIUnit\kgfe{\kg_{FeTi}}
\DeclareSIUnit\normvol{\cubic\meter_{i.N.}}
\DeclareSIUnit\normvolL{\liter_{i.N.}}

%%%%%%% New complete Commands

\NewDocumentCommand\DeclareNewQuantity{mmm}{%
	\DeclareSIUnit{#2}{#3}%
	\DeclareDocumentCommand{#1}{O{}m}{\SI[##1]{##2}{#2}}%
}

\DeclareNewQuantity
	\Beladungsdichte
	\beladungsdichte
	{\kgh\per\kgfe}
\DeclareNewQuantity
	\Dichte
	\dichte
	{\kg\per\cubic\meter}
\DeclareNewQuantity
	\Faraday
	\faraday
	{\ampere\second\per\mol}
\DeclareNewQuantity
	\Molarvolumen
	\molarvolumen
	{\liter\per\mol}
\DeclareNewQuantity
	\Normvolumen
	\normvolumen
	{\normvol}
\DeclareNewQuantity
	\Normvolumenstrom
	\normvolumenstrom
	{\normvolL\per\minute}
\DeclareNewQuantity
	\Normvolumenstromsec
	\normvolumenstromsec
	{\normvolL\per\second}
\DeclareNewQuantity
	\Volumenstrom
	\volumenstrom
	{\liter\per\minute}