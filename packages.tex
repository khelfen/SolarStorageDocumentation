%%%%%%%%% Dokumentenklasse
\documentclass[
	fontsize=12pt,
	paper=a4,
	oneside,
	leqno
	]{scrartcl}


%%%%%%%%% Sprachpackages
\usepackage[T1]{fontenc}								% Textzeichen für westeuropäische Sprachen
\usepackage[utf8]{inputenc}							% Umlaute
\usepackage[ngerman]{babel}						% Automatisch erzeugte Texte werden auf Deutsch ausgegeben
\usepackage{eurosym}									% €-Symbol


%%%%%%%%% Mathe und Naturwissenschaftliche Darstellung
\usepackage[
%	fleqn															% linksbündige Formeln
	]{amsmath}												% Vielzahl von neuen Umgebungen und Befehlen für den mathematischen Bereich
\usepackage{siunitx}									% Saubere Darestellung von SI Einheiten
\sisetup{
	locale = DE,												% Deutsche Norm, Kommas werden z.B. erkannt
	per-mode=fraction,									% Output a/b as \frac{a}{b} - in der Einheit
	quotient-mode=fraction,								% Output a/b as \frac{a}{b} - im Quotienten
	fraction-function=\tfrac,
	range-units = single,									% Einheit nur hinter zweitem Wert
	range-phrase = {\text{~bis~}},
	sticky-per = true,										% \per bleibt bestehen für mehr als eine Einheit
	separate-uncertainty									% Standardabweichung
}
\newenvironment{conditions}						% New environment - Für saubere Darstellung gegebener Variablen
	{\par\vspace{\abovedisplayskip}\noindent\begin{tabular}{>{$}l<{$} @{${}={}$} l}}
	{\end{tabular}\par\vspace{\belowdisplayskip}}

%%%%%%%%% Schritart
\usepackage{mathptmx}								% Times New Roman


%%%%%%%%% Grafiken und Farben
\usepackage{graphicx}									% viele grafische Befehle, z.B. \scalebox
\usepackage{xcolor, colortbl}							% Farben und Farbpaletten
\usepackage[export]{adjustbox}						% Positionieren von Grafiken, left, right, center
\usepackage{float}										% Floating Bilder, H-Befehl
\usepackage{mwe} 										% Fuer Abbildungsverzeichnis
\usepackage[center]{caption} 						% Zentriert Bildunterschriften
\usepackage[labelfont={color=black,sf,it},
	font={color=black,footnotesize,it},
	labelsep=space
	]{caption}													% customise captions
\usepackage{subfigure}									%support for the manipulation and reference of small or ‘sub’ figures and tables within a single figure or table environment
\usepackage{fancybox}									% kann Boxen (Rahmen) um Elemente legen


%%%%%%%%% Seitenformatierung
\usepackage[
	headsepline,												% Vertikale Linie unterm Header
	plainheadsepline,
	automark													% Section im Header
	]{scrlayer-scrpage}									% Paket zur Manipulation der Kopf- und Fußzeilen
\renewcommand{\headfont}{}						% Voreingestellte Schriftart bearbeiten, z.B. möglich \itshape für ein feines Kursiv
\usepackage[onehalfspacing]{setspace}			% Setzt Zeilenabstand auf 1.5
\usepackage[													% optische Verbesserungen des Dokuments
	final
	]{microtype}
\usepackage[
	cache=false,												% sonst gehts nicht^^
	newfloat,													% enables you to customize the listing environment
	]{minted}													% Für Programmiersprachen
% Zu minted: unter texmaker kofigurieren pdflatex: pdflatex -synctex=1 -interaction=nonstopmode --shell-escape %.tex
% Außerdem muss python 2.7+ und pygments installiert werden. Z.B. mit anaconda
\usepackage{caption}									% Für Captions von Code

% Environment um Code mit Captions darzustellen

\newenvironment{code}{\captionsetup{type=listing}}{}
\SetupFloatingEnvironment{listing}{name=Programmcode}

%%%%%%%%% Tabellenumgebung
\usepackage{booktabs}									% enhances the quality of tables
\usepackage{array}										% extends the options for column formats
\usepackage{multirow}									% mehrzeilige Tabellenzellen
\usepackage{tabularx}									% Tabellenumgebung, praktisch für Textweite-Tabellen, Standard Column-Type: X
\newcolumntype{R}{>{\raggedleft\arraybackslash}X}	% Neuer rechtsbündiger Column-Type


%%%%%%%%% Datumsformat \today
\usepackage[ngerman, num]{isodate}				% \today im DD. MM. YYYY Format
\daymonthsepgerman{}{}								%
\monthyearsepgerman{}{}								% entfernt Leerzeichen nach den Punkten


%%%%%%%%% Kopf- und Fußzeile
\ofoot*{\today}
\cfoot{Seite \pagemark}									% Mittleres Foot-Element

%%%%%%%%% Abkürzungsverzeichnis:
%\usepackage[													% Paket für Glossaries und Acronym-Glossaries
%	acronym,
%	automake,
%	nopostdot
%	]{glossaries}

%%%%%%%%% Quellen
\usepackage[
	backend = biber,
	bibencoding = utf8,
	style = alphabetic,
	block = space,
	]{biblatex}
	

%%%%%%%%% Formatierung TOC
\usepackage{tocstyle}
\newtocstyle[KOMAlike][leaders]{alldotted}{}	% Punktlinienführung, wie Abbildungs- und Tabellenverzeichnis
\usetocstyle{alldotted}

	
%\usepackage{tocloft}
%\setlength\cftparskip{0pt}
%\setlength\cftbeforesecskip{0pt}
%\setlength\cftaftertoctitleskip{1pt}

\usepackage[													% Zum Schluss laden, wegen Komplikationen
	hidelinks													% keine Boxen um die Links im PDF
	]{hyperref}												% Zulassen von Links u.ä. im PDF File